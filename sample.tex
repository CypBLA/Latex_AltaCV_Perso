%%%%%%%%%%%%%%%%%
% This is an sample CV template created using altacv.cls
% (v1.3, 10 May 2020) written by LianTze Lim (liantze@gmail.com). Now compiles with pdfLaTeX, XeLaTeX and LuaLaTeX.
%
%% It may be distributed and/or modified under the
%% conditions of the LaTeX Project Public License, either version 1.3
%% of this license or (at your option) any later version.
%% The latest version of this license is in
%%    http://www.latex-project.org/lppl.txt
%% and version 1.3 or later is part of all distributions of LaTeX
%% version 2003/12/01 or later.
%%%%%%%%%%%%%%%%

%% If you are using \orcid or academicons
%% icons, make sure you have the academicons
%% option here, and compile with XeLaTeX
%% or LuaLaTeX.
% \documentclass[10pt,a4paper,academicons]{altacv}

%% Use the "normalphoto" option if you want a normal photo instead of cropped to a circle
% \documentclass[10pt,a4paper,normalphoto]{altacv}

%%%%% CONFIGURATIONS INITIALES

\PassOptionsToPackage{svgnames}{xcolor}
% Sur conseil de : https://tex.stackexchange.com/questions/51488/option-clash-with-xcolor-and-tikz

%%%% Appel à la classe utilisée avec les paramètres désirés.
\documentclass[10pt,a4paper,ragged2e,withhyper,normalphoto]{altacv}
%% AltaCV uses the fontawesome5 and academicons fonts
%% and packages.
%% See http://texdoc.net/pkg/fontawesome5 and http://texdoc.net/pkg/academicons for full list of symbols. You MUST compile with XeLaTeX or LuaLaTeX if you want to use academicons.

%%%% Appel aux paquets supplémentaires nécessaires (non-inclus dans la classe)

% The paracol package lets you typeset columns of text in parallel
\usepackage{paracol}

% Opensans package
\usepackage[default,oldstyle,scale=1]{opensans}
% https://www.ctan.org/pkg/opensans

% Merriweather Sans
\usepackage[sf, scale=1]{merriweather}
% https://www.ctan.org/tex-archive/fonts/merriweather/
% https://tug.org/FontCatalogue/merriweathersans/
%
% [sf] to activate MerriweatherSans (sans serif) without Merriweather (serif)
% Commands \merriweather and \merriweathersans select the Merriweather and MerriweatherSans families, respectively
% Commands \merriweatherlight, \merriweatherblack, \merriweathersanslight, and \merriweathersansblack allow for localized use of light or black variants.

%\usepackage[sfdefault]{roboto}
% \usepackage[defaultsans]{lato}

\ExplSyntaxOn
	
	%%%% Paramétrage des dimensions
	% Change the page layout if you need to
	\geometry{left=0.5cm,right=0.5cm,top=0.5cm,bottom=0.5cm,columnsep=1.2cm}
	
	%%%% Options des liens et du PDF
	\hypersetup{
		pdftitle={CV - Cyprien  BLANC},
		pdfauthor={Cyprien BLANC}
	}
	
	%%%% Parametrage des fonts, polices, etc. 
	% CypBLA : Finalement pas besoin car définies à l'aide de l'appel aux paquets !
	
	% \usepackage[OT1]{fontenc}
	% Nécessaire ?
	
	% \renewcommand{\familydefault}{\sfdefault}
	% Normalement plus nécessaire comme OpenSans déclarée comme "default"
	
	%% CypBLA : Commenté car je ne vois pas de raisons de faire une distinction conditionnelle...
	%% Change the font if you want to, depending on whether
	%% you're using pdflatex or xelatex/lualatex
	%\ifxetexorluatex
	%  % If using xelatex or lualatex:
	%  \setmainfont{Roboto Slab}
	%  \setsansfont{Lato}
	%  \renewcommand{\familydefault}{\sfdefault}
	%\else
	%  % If using pdflatex:
	%  \usepackage[rm]{roboto}
	%  \usepackage[defaultsans]{lato}
	%  % \usepackage{sourcesanspro}
	%  \renewcommand{\familydefault}{\sfdefault}
	%\fi
	
	\renewcommand{\titlesfont}[1]{
		\textbf{\merriweathersans{#1}}
	}
	
	% Parametrage des differents tailles
	
	% Rappel de l'échelle des tailles :
	% \tiny	\scriptsize	\footnotesize	\small	\normalsize	\large	\Large	\LARGE	\huge	\Huge
	
	%% Change some fonts, if necessary
	\renewcommand{\namefont}[1]{
		\color{name}{\Huge{\titlesfont{#1}}}
	}
	
	\renewcommand{\taglinefont}[1]{
		\color{tagline}{\large{\titlesfont{#1}}}
	}
	
	\renewcommand{\personalinfofont}[1]{
		\footnotesize{\titlesfont{#1}}
	}
	
	\renewcommand{\cvsectionfont}[1]{
		\underline{\LARGE{\titlesfont{#1}}}
	}
	
	\renewcommand{\cvsubsectionfont}[1]{
		\underline{\large{\titlesfont{#1}}}
	}
	
	\renewcommand{\cvsubsubsectionfont}[1]{
		\underline{\normalsize{\titlesfont{#1}}}
	}
	
	%% Change the bullets for itemize and rating marker for \cvskill if you want to
	\renewcommand{\itemmarker}{{\small\textbullet}}
	\renewcommand{\ratingmarker}{\faCircle}
	
	%%%% Personnalisation des couleurs
	
	%% Définition de noms de couleurs
	
	% Outils utiles :
	% * HTML Color Values : https://www.w3schools.com/colors/colors_hex.asp
	% * Color picker : https://www.w3schools.com/colors/colors_picker.asp
	
	% A noter : 
	% Withinxcolor.sty, the following color names are defined:
	% red,green,blue,cyan,magenta,yellow,black,gray,white,darkgray,lightgray,
	% brown,lime,olive,orange,pink,purple,teal,violet.
	%
	% Option "svgnames" activée dans le paquet "xcolor"
	% De nombreuses teintes de couleur par nom sont donc déjà théoriquement définies.
	% Toutes celles au lien suivant : https://www.w3schools.com/colors/colors_names.asp
	
	%% Teintes grises
	%\definecolor{SlateGrey}{HTML}{2E2E2E} % Gris sombre
	%\definecolor{LightGrey}{HTML}{666666} % Gris moyen
	%\definecolor{Silver}{C0C0C0} % Gris clair
	%
	%% Teintes bleues
	%\definecolor{DeepSkyBlue}{00BFFF}
	%\definecolor{DodgerBlue}{1E90FF}
	%
	%% Teintes vertes
	%\definecolor{MediumSpringGreen}{00FA9A}
	%\definecolor{ForestGreen}{228B22}
	%\definecolor{SeaGreen}{2E8B57}
	%\definecolor{LimeGreen}{32CD32}
	%
	%% Teintes rouges
	%\definecolor{DarkPastelRed}{HTML}{450808} % Rouge très sombre presque marron
	%\definecolor{PastelRed}{HTML}{8F0D0D} % Rouge foncé
	%\definecolor{Tomato}{FF6347} % Rouge pastel
	%\definecolor{Coral}{FF7F50} % Corail 
	%
	%% Teintes jaunes
	%\definecolor{GoldenEarth}{HTML}{E7D192} % Couleur or/beige
	
	
	%% Définition des champs de couleurs utilisés dans le document.
	% Change the colours if you want to.
	\colorlet{name}{black} % Used for : \name
	\colorlet{tagline}{RoyalBlue} % Used for : \tagline
	\colorlet{heading}{RoyalBlue} % Used for : \cvsection
	\colorlet{headingrule}{black} % Used for : \rule
	\colorlet{subheading}{Navy} % Used for : \cvsubsection, \cvsubsubsection
	\colorlet{accent}{MediumBlue} % Used for : \cvref(#2), \cvskill(#2), \skillfive, \cvachievment(#1), \cvevent(#2), \quote, \printinfo(#1)
	\colorlet{emphasis}{MidnightBlue} % Used for : \cvref(#1), \cvskill(#1) \cvachievment(#2), \cvevent(#1)
	\colorlet{body}{black} % Used for : all text color
	
	%%%% Définitions de l'emplacement du fichier .bib de bibliographie
	%% sample.bib contains your publications
	\addbibresource{sample.bib}
	
	%%%% DEFINITION DE NOUVELLES COMMANDES
	\newcommand{\emphasis}[1]{\bfseries\textcolor{emphasis}{#1}}

\ExplSyntaxOff

%%%%% CONTENU
\begin{document}
	
%%%% Bandeau haut
\name{Cyprien BLANC}
\tagline{Jeune ingénieur en automobile et transport}
%% You can add multiple photos on the left or right
\photoR{2.8cm}{CypBLA_Square_500x500}
% \photoL{2.5cm}{Yacht_High,Suitcase_High}

\personalinfo{%
	% Not all of these are required!
	\email{cyprien.blanc@zaclys.net}
	\phone{(+33)6 98 13 97 81}
	\mailaddress{19 allée George Sand 69330 Jonage}
	\location{FRANCE}
	\linkedin{cyprienblanc}
	\github{CypBLA}
	% \homepage{<url>} % Espace Scenari à remplir : https://cyprienblanc.scenari-community.org/
	% \twitter{<twitterName>}
	%% You MUST add the academicons option to \documentclass, then compile with LuaLaTeX or XeLaTeX, if you want to use \orcid or other academicons commands.
	% \orcid{0000-0000-0000-0000}
	%
	%% You can add your own arbtrary detail with
	%% \printinfo{symbol}{detail}[optional hyperlink prefix]
	% \printinfo{\faPaw}{Hey ho!}[https://example.com/]
	%
	%% Or you can declare your own field with
	%% \NewInfoFiled{fieldname}{symbol}[optional hyperlink prefix] and use it:
	% \NewInfoField{gitlab}{\faGitlab}[https://gitlab.com/]
	% \gitlab{your_id}
}

\makecvheader
%% Depending on your tastes, you may want to make fonts of itemize environments slightly smaller
% \AtBeginEnvironment{itemize}{\small}

%%%% CONTENU EN COLONNES

%% Set the left/right column width ratio to 6:4.
\columnratio{0.7}

% Start a 2-column paracol. Both the left and right columns will automatically
% break across pages if things get too long.
\begin{paracol}{2}

	%%% FORMATION
	\cvsection{Formation}
	
	\framebox{
		\begin{minipage}{\linewidth}
			cvsection\cvsection{test}\par
			cvsectionfont\cvsectionfont{test}\par
			headingRule\headingRule\par
		\end{minipage}
	}
	
	\cvsectionfont{test}
	
	\cvevent{Diplôme universitaire - Formation Adaptée Enseignement (FAE) 1\textsuperscript{er} degré}{Institut National Supérieur du Professorat et de l'Education}{2019--2020}{Lyon}
	
	\begin{itemize}
		\item Formation à mi-temps alternée avec la prise en charge d'une classe de CP.
	\end{itemize}
	
	\divider
	
	\cvevent{Master 1 - Métiers de l'enseignement, de l'Education et de la Formation - Parcours de professeur des écoles}{Institut National Supérieur du Professorat et de l'Education}{2018--2019}{Lyon}
	Préparation parallèle du Concours de Recrutement des Professeurs des Ecoles (CRPE)
	
	\divider
	
	\cvevent{Cursus ingénieur complet}{ISAT}{2013--2018}{ISAT}
	\begin{itemize}
		\item Département EPEE (\'{E}nergétique, Propulsion, \'{E}lectronique et Environnement) - Spécialisation VIA (Véhicule Intelligent et Autonome)
		\item Cursus en cinq ans avec deux stages ingénieur de six mois et un semestre à l'étranger
	\end{itemize}
	
	\divider
	
	\cvevent{Semestre d'études Erasmus - Automation and Control Engineering}{Politecnico di Milano}{03/17--07/17}{Milan - Italie}
	\begin{itemize}
		\item Cours en langue anglaise et quotidien en italien
	\end{itemize}
	
	\divider
	
	\cvevent{Terminale Scientifique - Spé. Sciences de l'ingénieur}{Lycée Charlie Chaplin}{2012--2013}{Décines-Charpieu}
	\begin{itemize}
		\item Baccalauréat mention Très Bien (16,55/20)
		\item Spécialité Sciences de l'Ingénieur
		\item Option musique
	\end{itemize}
	
	%\cvevent{Titre}{Entreprise}{Année}{Lieu}
	%\begin{itemize}
	%\item
	%\item
	%\end{itemize}
		
	\medskip
	
	%%% EXPERIENCES
	\cvsection{Expériences}
	
	%% PROFESSIONNELLES
	\cvsubsection{Professionnelles}
	
	%\cvevent{Titre}{Entreprise}{Année}{Lieu}
	%\begin{itemize}
	%\item
	%\item
	%\end{itemize}
	
	\cvevent{Professeur des écoles stagiaire}{Education nationale}{2019--2020}{Soucieux-en-Jarrest}
	\begin{itemize}
		\item En charge les lundi et mardi d'une classe de CP à l'école élémentaire "Les Chadrillons" de Soucieu-en-Jarrest
		\item En formation en parallèle à mi-temps à l'Institut National Supérieur du Professorat et de l'Education de Lyon
	\end{itemize}
	
	\divider
	
	\cvevent{Adjoint territorial d'animation}{Mairie de Meyzieu}{07/19}{Meyzieu}
	\begin{itemize}
		\item Trois séjours campés de 5 jours sous tente
		\item Encadrement de jeunes de 12 à 16 ans
	\end{itemize}
	
	\divider
	
	\cvevent{Stage ingénieur - Développement Matlab - Logiciel de post-traitement de données d'essais moteurs}{MCE-5}{01/18--06/18}{Lyon}
	\begin{itemize}
		\item Intégré au sein de l'équipe calcul
		\item Post-traitement de données d'essais moteurs à architecture innovante (taux de compression variable)
	\end{itemize}
	
	\divider
	
	\cvevent{Stagé ingénieur - Développement Matlab - Outils d'analyse de données d'essais}{Valéo}{07/16--12/16}{Cergy}
	\begin{itemize}
		\item Intégré au sein de l'équipe système autour du produit "Electric SuperCharger", un compresseur de suralimentation électrique
		\item Intervention sur deux outils à interfaces graphiques :
		\begin{itemize}
			\item Outil d'importation de données, de visualisation de signaux et d'export de graphiques
			\item Outil de traitement des campagnes de tests et d'acquisition du produit \emph{electric SuperCharger} 
		\end{itemize}
	\end{itemize}
	
	\divider
	
	\cvevent{Missions ponctuelles d'intérim}{Agences d'intérim}{2014 {\textasciitilde{}} 2018}{Lyon}
	\begin{itemize}
		\item Manutentionnaire
		\item Monteur-câbleur
		\item Animateur
	\end{itemize}
	
	\medskip
	
	%% SCOLAIRES
	\cvsubsection{Scolaires}
	
	\cvevent{Projets d'ingénierie}{ISAT}{2015--2017}{Nevers}
	\begin{itemize}
		\item ISAT Hydrogen Project\\%
		Modélisation et simulation d'un véhicule hybride série à pile à combustible sous Matlab/Simulink
		\item Projet multi-disciplinaire - Défi de parcours semi-autonome\\%
		Véhicule de modélisme modifié asservi au travers d'une communication distante depuis PC. Aucune action humaine au cours même du parcours.
		\item Modèle de moteur à combustion interne 0D 1 zone\\%
		Modèle développé sous Matlab.
	\end{itemize}
	
	%\cvevent{Titre}{Entreprise}{Année}{Lieu}
	%\begin{itemize}
	%	\item
	%	\item
	%\end{itemize}
	
	\medskip
	
	%% ASSOCIATIVES ET BENEVOLES
	\cvsubsection{Associatives et bénévoles}
	
	\cvevent{Encadrement de camp scout}{Scouts et Guides de France}{07/18}{Sembadel Gare}
	\begin{itemize}
		\item Encadrement de jeunes de 11 à 14 ans
		\item Coopération entre encadrants sur les actions en amont et en cours de camp
		\item Planification d'activités, logistique, encadrement, etc.
	\end{itemize}
	
	\cvevent{Association ISATevent}{ISAT}{2016--2018}{Nevers}
	Organisation du gala de l'ISAT : gestion de billetterie, définitions des plans de table, recherche et démarchage de groupes de musique, etc.
	
	\cvevent{Vice-président du Bureau des \'{E}lèves}{ISAT}{2015--2016}{Nevers}
	Organisation d'évènements, gestion d'un local associatif, réalisation de supports graphiques, démarchage, etc.
	
	\cvevent{Représentant étudiant - Conseil d'administration}{ISAT}{2016--2018}{Nevers}
	
	%%% COMPETENCES
	\cvsection{Compétences}
	
	%% Informatique
	\cvsubsection{Informatique}
	
	% LANGAGES MODELISATION ET SIMULATION
	\cvsubsubsection{Langages de modélisation et simulation}
	\cvtag{Modelica}
	\cvtag{Simulink}
	
	% PROGRAMMATION
	\cvsubsubsection{Programmation}
	\cvtag{Programmation Orientée Objets}
	\cvtag{Matlab}
	\cvtag{Labview}
	\cvtag{VBA (Excel)}
	\cvtag{Langages web (HTML, CSS, PHP, SQL)}
	\cvtag{C}
	\cvtag{Python}
	
	% SYSTEMES D'EXPLOITATION
	\cvsubsubsection{Systèmes d'exploitation}
	\cvtag{Windows}
	\cvtag{GNU/Linux}
	
	% OUTILS BUREAUTIQUES
	\cvsubsubsection{Outils bureautiques}
	\begin{itemize}
		\item Suites office : 
			\cvtag{Microsoft Office}
			\cvtag{LibreOffice}
		\item Edition multimedia (images, vidéo, audio, graphiques, etc.)
	\end{itemize}
	
	%%% COMPETENCES
	\cvsubsection{Langues}
	
	\begin{tabularx}{\linewidth}{ X X }
		\emphasis{Français} & \skillfive{5} \\ 
		\hline
		\emphasis{Anglais} & \skillfive{4}\par%
		\begin{itemize}
				\item Niveau CERCL B2-C1
				\item TOEIC 820pts (\faCalendar~2014)
				\item Lecture et visionnage régulier de média en en langue anglais.
		\end{itemize} \\ 
		\hline
		\emphasis{Italien} & \skillfive{2}\par%
		Niveau CERCL B1 \\ 
	\end{tabularx}
	
	%%% CENTRES D'INTERETS
	\cvsection{Centres d'intérêts}
	
	%% Pratiques sportives
	\cvsubsection{Pratiques sportives}
	\begin{itemize}
		\item[\faAngleDoubleUp] Escalade en club\par%
		\begin{quote}
			Me dépasser et affronter des obstacles toujours plus difficiles en sachant pouvoir toujours compter sur mon binôme pour m'assurer et me prodiguer des conseils.
		\end{quote}
		\item[\faBiking] Vélo\par%
		\begin{quote}
			Me dépenser et braver les éléments tout en faisant un geste pour l'environnement. Savoir entretenir mon vélo et me remettre en selle.
		\end{quote}
	\end{itemize}
		
	\cvsubsection{\faWrench Mécanique}
	\begin{itemize}
		\item Mécanique vélo
		\item Réparations et entretiens divers
	\end{itemize}
	
	\cvsubsection{\faLaptopCode Informatique}
	\begin{itemize}
		\item Monde du libre
		\item Programmation
		\item Administration informatique
	\end{itemize}
		
	% use ONLY \newpage if you want to force a page break for
	% ONLY the current column
	\newpage

\end{paracol}

\end{document}
