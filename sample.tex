%%%%%%%%%%%%%%%%%
% This is an sample CV template created using altacv.cls
% (v1.3, 10 May 2020) written by LianTze Lim (liantze@gmail.com). Now compiles with pdfLaTeX, XeLaTeX and LuaLaTeX.
%
%% It may be distributed and/or modified under the
%% conditions of the LaTeX Project Public License, either version 1.3
%% of this license or (at your option) any later version.
%% The latest version of this license is in
%%    http://www.latex-project.org/lppl.txt
%% and version 1.3 or later is part of all distributions of LaTeX
%% version 2003/12/01 or later.
%%%%%%%%%%%%%%%%

%% If you are using \orcid or academicons
%% icons, make sure you have the academicons
%% option here, and compile with XeLaTeX
%% or LuaLaTeX.
% \documentclass[10pt,a4paper,academicons]{altacv}

%% Use the "normalphoto" option if you want a normal photo instead of cropped to a circle
% \documentclass[10pt,a4paper,normalphoto]{altacv}

%%%%% CONFIGURATIONS INITIALES

\PassOptionsToPackage{svgnames}{xcolor}
% Sur conseil de : https://tex.stackexchange.com/questions/51488/option-clash-with-xcolor-and-tikz

%%%% Appel à la classe utilisée avec les paramètres désirés.
\documentclass[8pt,a4paper,ragged2e,withhyper,normalphoto]{altacv}
%% AltaCV uses the fontawesome5 and academicons fonts
%% and packages.
%% See http://texdoc.net/pkg/fontawesome5 and http://texdoc.net/pkg/academicons for full list of symbols. You MUST compile with XeLaTeX or LuaLaTeX if you want to use academicons.

%%%% Appel aux paquets supplémentaires nécessaires (non-inclus dans la classe)

% Opensans package
\usepackage[default,oldstyle,scale=1]{opensans}
% https://www.ctan.org/pkg/opensans

% Merriweather Sans
\usepackage[sf, scale=1]{merriweather}
% https://www.ctan.org/tex-archive/fonts/merriweather/
% https://tug.org/FontCatalogue/merriweathersans/
%
% [sf] to activate MerriweatherSans (sans serif) without Merriweather (serif)
% Commands \merriweather and \merriweathersans select the Merriweather and MerriweatherSans families, respectively
% Commands \merriweatherlight, \merriweatherblack, \merriweathersanslight, and \merriweathersansblack allow for localized use of light or black variants.

\ExplSyntaxOn
	
	%%%% Paramétrage des dimensions
	% Change the page layout if you need to
	\geometry{left=0.5cm,right=0.5cm,top=0.5cm,bottom=0.5cm,columnsep=1.2cm}
	
	%%%% Options for links and pdf
	\hypersetup{
		pdftitle={CV - Cyprien  BLANC},
		pdfauthor={Cyprien BLANC}
	}
	
	%%%% 
	
	%% Command applied for every element we can consider as a kind of title
	% Put in bold and apply the specific font.
	\renewcommand{\titlesfont}[1]{
		\textbf{\merriweathersans{#1}}
	}
	
	% Rappel de l'échelle des tailles :
	% \tiny	\scriptsize	\footnotesize	\small	\normalsize	\large	\Large	\LARGE	\huge	\Huge
	
	%%%% Heading macros
	%% Name configuration
	\renewcommand{\nameTitle}[1]{
		\color{name}{\Huge{\titlesfont{#1}}}\par
		\vspace{0mm}
	}

	%% Tagline configuration (under the name)
	\renewcommand{\taglinePar}[1]{
		\color{tagline}{\Large{\titlesfont{#1}}}\par
		\smallskip
	}
	
	%% Configuration of the personal information listing look (made with \printinfo)
	\renewcommand{\personalInfoPar}[1]{
		\begingroup
			\linespread{1.3}
			\large{\titlesfont{#1}}\par
		\endgroup
	}
	
	%% Icons color in the heading.
	\renewcommand{\infoIconFont}[1]{
		\color{accent}{#1}
	}
	
	%% Spacing between personnal info fields in the head
	\renewcommand{\interIconsHSpace}{
		\hspace{1.8em}
	}
	
	%%%% CV sections macros
	
	%% Thick line put under the cv section title
	\renewcommand{\headingRule}{
		\color{headingrule}{\rule{\linewidth}{1pt}}
	}
	
	%% vspace applied on every section title
	\newcommand{\titleSkip}{
		\vspace{1.5pt}
	}
	
	%% Constant dimension used as reference for titles indentation.
	\dim_const:Nn \c_titlesIndentRef_dim {1ex}
	
	%% Command \cvsection
	% > \cvsection{heading}
	\renewcommand{\cvsection}[1]{
		\titleSkip
		\color{heading}{\LARGE{\titlesfont{#1}}}\\
		\vspace{-1.5ex}
		\headingRule\par
		\titleSkip
	}
	
	%% Command \cvsubsection
	% > \cvsubsection{heading}
	\renewcommand{\cvsubsection}[1]{
		\titleSkip
		\hspace{
			\fp_eval:n {1 * \c_titlesIndentRef_dim} pt
		}
		\color{subheading}{\uline{\large{\titlesfont{#1}}}}\par
		\titleSkip
	}

	%% Command \cvsubsubsection
	% > cvsubsubsection{heading}
	\renewcommand{\cvsubsubsection}[1]{
		\titleSkip
		\hspace{
			\fp_eval:n {2 * \c_titlesIndentRef_dim} pt
		}
		\color{subsubheading}{\uline{\normalsize{\titlesfont{#1}}}}\par
		\titleSkip
	}

	%%%% CV event configuration
	
	\renewcommand{\cveventTitle}[1]{
		\color{emphasis}{\large{#1}}\par
	}
	
	\renewcommand{\cveventPlacePar}[1]{
		\color{emphasis}{\textbf{#1}}\par
	}
	
	\renewcommand{\cveventCalendarBox}[1]{
		\makebox[0.5\linewidth][l]{
			\small{\faCalendar~#1}
		}
	}
	
	\renewcommand{\cveventMapBox}[1]{
		\makebox[0.5\linewidth][l]{
			\small{\faMapMarker~#1}
		}
	}
	
	%%%% 

	% Used in :
	% * \cvskill
	% * \cvref
	\renewcommand{\textEmph}[1]{
		\color{emphasis}{\textbf{#1}}
	}
	
	%%%%
	
	%% Divider config
	% Separator placed between elements as desired
	\renewcommand{\divider}{
		\vspace{-0.5em}
		\color{body!30}{\hdashrule{\linewidth}{0.6pt}{0.5ex}}
		\titleSkip
	}
	
	\renewcommand{\ratingmarker}{\faCircle}
	
	%%%% 

	%% Environment \begin{quote}
	% Adaptation of the quote environment
	\renewenvironment{quote}{\color{accent}\itshape}{\par}
	
	%%%% Lists configuration
	
	%% Change the bullets for itemize and rating marker for \cvskill if you want to
	\renewcommand{\itemmarker}{{\small\textbullet}}
	
	\setlist{
		% Horizontal distances
		labelindent=0ex,% Space from left margin to left of label box
		labelwidth=1ex,% Width of label box
		labelsep=0.5ex,% Space from right of label box to item paragraph first line beginning
		itemindent=0ex,% Indentation of the first line of item paragraph compared to the following lines
		leftmargin=!,% Distance from left margin to item paragraph (without item indent)
		% Vertical distances
		topsep=0pt,
		itemsep=0pt,
		parsep=0.1\baselineskip
	}
	% Enumitem symbols :
	% "!" : Sets which value is to be computed from the others.
	% "*" : 
	% itemindent=*, sets the minimal width to that of widest label
	
	% Solutions found there to avoid spaces in itemize : 
	% https://stackoverflow.com/questions/3275622/latex-remove-spaces-between-items-in-list
	
	\setlist[itemize]{label=\itemmarker}
	
	%% Personnalisation espace entre colonnes dans environnement multicol
	\setlength{\columnsep}{0.5cm}
	
	\setlength{\columnseprule}{1pt}
	\def\columnseprulecolor{\color{black}}
	
	%%%% Table display aeration
%	See this article : https://texblog.org/2017/02/06/proper-tables-with-latex/
%	\renewcommand\arraystretch{1.3}
	\setlength{\extrarowheight}{2pt}
		
%	\renewcommand{\tabcolsep}{3pt}
	
	%%%% Colors personalisation
	
	% Outils utiles :
	% * HTML Color Values : https://www.w3schools.com/colors/colors_hex.asp
	% * Color picker : https://www.w3schools.com/colors/colors_picker.asp
	
	% A noter : 
	% * Within xcolor.sty, the following color names are defined:
	% red,green,blue,cyan,magenta,yellow,black,gray,white,darkgray,lightgray,
	% brown,lime,olive,orange,pink,purple,teal,violet.
	%
	% * "svgnames" option activated in the "xcolor" package
	% De nombreuses teintes de couleur par nom sont donc déjà théoriquement définies.
	% Toutes celles au lien suivant : https://www.w3schools.com/colors/colors_names.asp
	
	%	xcolor color models : natural, rgb, cmy, cmyk, hsb, gray, RGB, HTML, HSB, Gray
	%	* \color[rgb]{.1,.2,.3})
	%   * \color[Gray]{15}
	
	% Available options
	\str_const:Nn \l_greyColorHue_str {greyHue}
	\str_const:Nn \l_blueColorHue_str {blueHue}
	\str_const:Nn \l_goldenAndRed_str {goldenAndRed}
	
	% To modify to choose the switch-case option
	\str_set_eq:NN \l_colorHueSelection_str \l_greyColorHue_str
	
	% String to verify if the switch case worked
	\str_new:N \l_effectiveColorHue_str
	
	%% Color fields definition
	\str_case_e:nn {\l_colorHueSelection_str}
	{
		{\l_greyColorHue_str} {
			\colorlet{name}{black} % Used for : \name
			\colorlet{tagline}{black!85} % Used for : \tagline
			\colorlet{heading}{black!85} % Used for : \cvsection
			\colorlet{headingrule}{black} % Used for : \rule
			\colorlet{subheading}{black!70} % Used for : \cvsubsection, \cvsubsubsection
			\colorlet{accent}{black} % Used for : \cvref(#1), \cvskill(#1), \skillfive, \cvachievment(#1), \cvevent(#1), \quote, \printinfo(#1)
			\colorlet{emphasis}{black} % Used for : \cvref(#1), \cvskill(#1) \cvachievment(#2), \cvevent(#1)
			\colorlet{body}{black} % Used for : all text color
			
%			\str_set_eq:NN \l_effectiveColorHue_str \l_greyColorHue_str
		}
	
		{\l_blueColorHue_str} {
			%% A AMELIORER %%
			\colorlet{name}{black} % Used for : \name
			\colorlet{tagline}{RoyalBlue} % Used for : \tagline
			\colorlet{heading}{RoyalBlue} % Used for : \cvsection
			\colorlet{headingrule}{black} % Used for : \rule
			\colorlet{subheading}{Navy} % Used for : \cvsubsection, \cvsubsubsection
			\colorlet{accent}{MediumBlue} % Used for : \cvref(#2), \cvskill(#2), \skillfive, \cvachievment(#1), \cvevent(#2), \quote, \printinfo(#1)
			\colorlet{emphasis}{MidnightBlue} % Used for : \cvref(#1), \cvskill(#1) \cvachievment(#2), \cvevent(#1)
			\colorlet{body}{black} % Used for : all text color

%			\str_set_eq:NN \l_effectiveColorHue_str \l_blueColorHue_str
		}
	
		{\l_goldenAndRed_str}{
			\definecolor{SlateGrey}{HTML}{2E2E2E}
			\definecolor{LightGrey}{HTML}{666666}
			\definecolor{DarkPastelRed}{HTML}{450808}
			\definecolor{PastelRed}{HTML}{8F0D0D}
			\definecolor{GoldenEarth}{HTML}{E7D192}
			\colorlet{name}{black}
			\colorlet{tagline}{PastelRed}
			\colorlet{heading}{DarkPastelRed}
			\colorlet{headingrule}{GoldenEarth}
			\colorlet{subheading}{PastelRed}
			\colorlet{subsubheading}{PastelRed}
			\colorlet{accent}{PastelRed}
			\colorlet{emphasis}{SlateGrey}
			\colorlet{body}{LightGrey}
		}
	}
	
	
%	\colorlet{name}{\color[Gray]{15}} % Used for : \name
%	\colorlet{tagline}{\color[Gray]{13}} % Used for : \tagline
%	\colorlet{heading}{\color[Gray]{13}} % Used for : \cvsection
%	\colorlet{headingrule}{\color[Gray]{15}} % Used for : \rule
%	\colorlet{subheading}{\color[Gray]{13}} % Used for : \cvsubsection, \cvsubsubsection
%	\colorlet{accent}{\color[Gray]{10}} % Used for : \cvref(#1), \cvskill(#1), \skillfive, \cvachievment(#1), \cvevent(#1), \quote, \printinfo(#1)
%	\colorlet{emphasis}{\color[Gray]{9}} % Used for : \cvref(#1), \cvskill(#1) \cvachievment(#2), \cvevent(#1)
%	\colorlet{body}{\color[Gray]{15}} % Used for : all text color
	
	%%%% Définitions de l'emplacement du fichier .bib de bibliographie
	%% sample.bib contains your publications
	\addbibresource{sample.bib}
	
	%%%% New commands definition
	% > \emphasis{<text>}
	% Used to emphase Language name in languages section.
	\newcommand{\emphasis}[1]{\color{emphasis}{\textbf{#1}}}
	
\ExplSyntaxOff

%%%%% CONTENU
\begin{document}
	
%%%% Bandeau haut
\name{Cyprien BLANC}
\tagline{Jeune ingénieur en automobile et transport}

%% You can add multiple photos on the left or right
\photoR{2.8cm}{CypBLA_Square_500x500}
% \photoL{2.5cm}{Yacht_High,Suitcase_High}

\personalinfo{%
	% Not all of these are required!
	\birth{24/05/1995, 25 ans}
	\phone{(+33)6 98 13 97 81}
	\email{cyprien.blanc@zaclys.net}
	\mailaddress{19 allée George Sand 69330 Jonage}
%	\location{FRANCE}
	\drivingLicence{Permis A et B -- véhicule personnel}
	\linkedin{cyprienblanc}
	\github{CypBLA}
	% \homepage{<url>} % Espace Scenari à remplir : https://cyprienblanc.scenari-community.org/
	% \twitter{<twitterName>}
	%% You MUST add the academicons option to \documentclass, then compile with LuaLaTeX or XeLaTeX, if you want to use \orcid or other academicons commands.
	% \orcid{0000-0000-0000-0000}
	%
	%% You can add your own arbtrary detail with
	%% \printinfo{symbol}{detail}[optional hyperlink prefix]
	% \printinfo{\faPaw}{Hey ho!}[https://example.com/]
	%
	%% Or you can declare your own field with
	%% \NewInfoFiled{fieldname}{symbol}[optional hyperlink prefix] and use it:
	% \NewInfoField{gitlab}{\faGitlab}[https://gitlab.com/]
	% \gitlab{your_id}
}

\makecvheader

%\ExplSyntaxOn
%	Selected~coloration~: \\
%	(~ \str_use:N \l_effectiveColorHue_str ~)
%\ExplSyntaxOff

%% Depending on your tastes, you may want to make fonts of itemize environments slightly smaller
% \AtBeginEnvironment{itemize}{\small}

%%%% CONTENU EN COLONNES

% Start a 2-column paracol. Both the left and right columns will automatically
% break across pages if things get too long.
%% Set the left/right column width ratio to 6:4.
%\columnratio{0.6}
%\begin{paracol}{2}
\begin{multicols*}{2}

	%%% FORMATION
	\cvsection{Formation}
	
	\cvevent{Diplôme universitaire - Formation Adaptée Enseignement (FAE) 1\textsuperscript{er} degré}{Institut National Supérieur du Professorat et de l'Education (INSPE)}{2019 -- 2020}{Lyon, Rhône}
	
	\begin{itemize}
%		\DrawEnumitemLabel
		\item Formation à mi-temps alternée avec la prise en charge d'une classe de CP.
	\end{itemize}
	
	\divider
	
	\cvevent{Master 1 - Métiers de l'enseignement, de l'Education et de la Formation - Parcours de professeur des écoles}{Institut National Supérieur du Professorat et de l'Education (INSPE)}{2018 -- 2019}{Lyon, Rhône}
	Préparation parallèle du Concours de Recrutement des Professeurs des Ecoles (CRPE)
	
	\divider
	
	\cvevent{Cursus ingénieur complet}{Institut Supérieur de l'Automobile et des Transports (ISAT)}{2013 -- 2018}{ISAT, Nièvre}
	\begin{itemize}
		\item Département EPEE (\'{E}nergétique, Propulsion, \'{E}lectronique et Environnement) - Spécialisation VIA (Véhicule Intelligent et Autonome)
		\item Cursus en cinq ans avec deux stages ingénieur de six mois et un semestre à l'étranger
	\end{itemize}
	
	\divider
	
	\cvevent{Semestre d'études Erasmus - Automation and Control Engineering}{Politecnico di Milano}{03/17 -- 07/17}{Milan - Italie}
	\begin{itemize}
		\item Cours en langue anglaise et quotidien en italien
	\end{itemize}
	
	\divider
	
	\cvevent{Terminale Scientifique - Spé. Sciences de l'ingénieur}{Lycée Charlie Chaplin}{2012 -- 2013}{Décines-Charpieu, Rhône}
	\begin{itemize}
		\item Baccalauréat mention Très Bien (16,55/20)
		\item Spécialité Sciences de l'Ingénieur
		\item Option musique
	\end{itemize}
	
	%\cvevent{Titre}{Entreprise}{Année}{Lieu}
	%\begin{itemize}
	%\item
	%\item
	%\end{itemize}

	%%% EXPERIENCES
	\cvsection{Expériences}
	
	%% PROFESSIONNELLES
	\cvsubsection{Professionnelles}
	
	%\cvevent{Titre}{Entreprise}{Année}{Lieu}
	%\begin{itemize}
	%\item
	%\item
	%\end{itemize}
	
	\cvevent{Professeur des écoles stagiaire}{Education nationale}{2019 -- 2020}{Soucieux-en-Jarrest, Rhône}
	\begin{itemize}
		\item En charge les lundi et mardi d'une classe de CP à l'école élémentaire "Les Chadrillons" de Soucieu-en-Jarrest
		\item En formation en parallèle à mi-temps à l'Institut National Supérieur du Professorat et de l'Education de Lyon
	\end{itemize}
	
	\divider
	
	\cvevent{Adjoint territorial d'animation}{Mairie de Meyzieu}{07/19}{Meyzieu, Rhône}
	\begin{itemize}
		\item Trois séjours campés de 5 jours sous tente
		\item Encadrement de jeunes de 12 à 16 ans
	\end{itemize}
	
	\divider
	
	\cvevent{Stage ingénieur - Développement Matlab - Logiciel de post-traitement de données d'essais moteurs}{MCE-5}{01/18 -- 06/18}{Lyon, Rhône}
	\begin{itemize}
		\item Intégré au sein de l'équipe calcul
		\item Post-traitement de données d'essais moteurs à architecture innovante (taux de compression variable)
	\end{itemize}
	
	\divider
	
	\cvevent{Stage ingénieur - Développement Matlab - Outils d'analyse de données d'essais}{Valéo}{07/16 -- 12/16}{Cergy, Val d'Oise}
	\begin{itemize}
		\item Intégré au sein de l'équipe système autour du produit "Electric SuperCharger", un compresseur de suralimentation électrique
		\item Intervention sur deux outils à interfaces graphiques :
		\begin{itemize}
			\item Outil d'importation de données, de visualisation de signaux et d'export de graphiques
			\item Outil de traitement des campagnes de tests et d'acquisition du produit \emph{electric SuperCharger} 
		\end{itemize}
	\end{itemize}
	
	\divider
	
	\cvevent{Missions ponctuelles d'intérim}{Agences d'intérim}{2014 {\textasciitilde{}} 2018}{Lyon, Rhône}
	\begin{itemize}
		\item Manutentionnaire
		\item Monteur-câbleur
		\item Animateur
	\end{itemize}
	
	%% SCOLAIRES
	\cvsubsection{Scolaires}
	
%	\cvevent{Projets d'ingénierie}{ISAT}{2015 -- 2017}{Nevers, Nièvre}
%	\begin{itemize}
%		\item ISAT Hydrogen Project\\%
%		Modélisation et simulation d'un véhicule hybride série à pile à combustible sous Matlab/Simulink
%		\item Projet multi-disciplinaire - Défi de parcours semi-autonome\\%
%		Véhicule de modélisme modifié asservi au travers d'une communication distante depuis PC. Aucune action humaine au cours même du parcours.
%		\item Modèle de moteur à combustion interne 0D 1 zone\\%
%		Modèle développé sous Matlab.
%	\end{itemize}
	
	\cveventTitle{Projets d'ingénierie ISAT}\par
	\cveventCalendarBox{2015 {\textasciitilde{}} 2017} \cveventMapBox{Nevers, Nièvre}\par
	\cveventPlacePar{ISAT Hydrogen Project}\par
	Modélisation et simulation d'un véhicule hybride série à pile à combustible sous Matlab/Simulink.
	
	\cveventPlacePar{Projet multi-disciplinaire - Défi de parcours semi-autonome}\par
	Véhicule de modélisme modifié asservi au travers d'une communication distante depuis PC. Aucune action humaine au cours même du parcours.
	
	\cveventPlacePar{Modèle de moteur à combustion interne 0D 1 zone}\par
	Modèle de calcul développé sous forme d'un script Matlab.
	
	%\cvevent{Titre}{Entreprise}{Année}{Lieu}
	%\begin{itemize}
	%	\item
	%	\item
	%\end{itemize}
	
	%% ASSOCIATIVES ET BENEVOLES
	\cvsubsection{Associatives et bénévoles}
	
	\cvevent{Encadrement de camp scout}{Scouts et Guides de France}{07/18}{Sembadel Gare, Haute-Loire}
	\begin{itemize}
		\item Encadrement de jeunes de 11 à 14 ans
		\item Coopération entre encadrants sur les actions en amont et en cours de camp
		\item Planification d'activités, logistique, encadrement, etc.
	\end{itemize}
	
	\cveventTitle{Associations ISAT}\par
	\cveventCalendarBox{2015 {\textasciitilde{}} 2018} \cveventMapBox{Nevers, Nièvre}\par
	\cveventPlacePar{Association ISATevent}\par
	Organisation du gala de l'ISAT : gestion de billetterie, définitions des plans de table, recherche et démarchage de groupes de musique, etc.
	
	\cveventPlacePar{Vice-président du Bureau des \'{E}lèves}\par
	Organisation d'évènements, gestion d'un local associatif, réalisation de supports graphiques, démarchage, etc.
	
	\cveventPlacePar{Représentant étudiant - Conseil d'administration}
	
%	\cvevent{Association ISATevent}{ISAT}{2016 -- 2018}{Nevers, Nièvre}
%	Organisation du gala de l'ISAT : gestion de billetterie, définitions des plans de table, recherche et démarchage de groupes de musique, etc.
%	
%	\cvevent{Vice-président du Bureau des \'{E}lèves}{ISAT}{2015 -- 2016}{Nevers, Nièvre}
%	Organisation d'évènements, gestion d'un local associatif, réalisation de supports graphiques, démarchage, etc.
%	
%	\cvevent{Représentant étudiant - Conseil d'administration}{ISAT}{2016 -- 2018}{Nevers, Nièvre}
	
	%%% COMPETENCES
	\cvsection{Compétences}
	
	%% Informatique
	\cvsubsection{\faLaptopCode Informatique}
	
	% LANGAGES MODELISATION ET SIMULATION
	\cvsubsubsection{Langages de modélisation et simulation}
	\cvtag{Modelica}
	\cvtag{Simulink}
	
	% PROGRAMMATION
	\cvsubsubsection{Programmation}
	\cvtag{Programmation Orientée Objets}
	\cvtag{Matlab}
	\cvtag{Labview}
	\cvtag{VBA (Excel)}
	\cvtag{Langages web (HTML, CSS, PHP, SQL)}
	\cvtag{C}
	\cvtag{Python}
	
	% SYSTEMES D'EXPLOITATION
	\cvsubsubsection{Systèmes d'exploitation}
	\cvtag{Windows}
	\cvtag{GNU/Linux}
	
	% OUTILS BUREAUTIQUES
	\cvsubsubsection{Outils bureautiques}
	\begin{itemize}
		\item Suites office : 
			\cvtag{Microsoft Office}
			\cvtag{LibreOffice}
		\item Edition multimedia (images, vidéo, audio, graphiques, etc.)
	\end{itemize}
	
	%%% COMPETENCES
	\cvsubsection{\faLanguage Langues}
	
	\begin{tabular}{ l l }
		\emphasis{Français}\par & \skillfive{5}\par \\ 
		\hline
		\emphasis{Anglais}\par & \skillfive{4}\par \\
%		\multicolumn{2}{l}{
%			\shortstack[l]{
%				\textbullet Niveau CERCL B2-C1 \\
%				\textbullet TOEIC 820pts (\faCalendar~2014) \\
%				\textbullet Lecture et visionnage régulier de média en en langue anglais. \\
%			}
%		} \\
		\multicolumn{2}{l}{\shortstack[l]{\textbullet Niveau CERCL B2-C1 \\ %
				\textbullet TOEIC 820pts (\faCalendar~2014) \\ %
				\textbullet Lecture et visionnage régulier de média en en langue anglais.}} \\
		\hline
		\emphasis{Italien}\par & \skillfive{2}\par \\
%		\multicolumn{2}{l}{
%			\textbullet Niveau CERCL B1
%		} \\
		\multicolumn{2}{l}{\textbullet Niveau CERCL B1} \\
	\end{tabular}
	
	%%% CENTRES D'INTERETS
	\cvsection{Centres d'intérêts}
	
	%% Pratiques sportives
	\cvsubsection{Pratiques sportives}
	\begin{itemize}
		\item[\faAngleDoubleUp] Escalade en club\par%
		\begin{quote}
			Me dépasser et affronter des obstacles toujours plus difficiles en sachant pouvoir toujours compter sur mon binôme pour m'assurer et me prodiguer des conseils.
		\end{quote}
		\item[\faBiking] Vélo\par%
		\begin{quote}
			Me dépenser et braver les éléments tout en faisant un geste pour l'environnement. Savoir entretenir mon vélo et me remettre en selle.
		\end{quote}
	\end{itemize}
		
	\cvsubsection{\faWrench Mécanique}
	\begin{itemize}
		\item Mécanique vélo
		\item Réparations et entretiens divers
	\end{itemize}
	
	\cvsubsection{\faLaptopCode Informatique}
	\begin{itemize}
		\item Monde du libre
		\item Programmation
	\end{itemize}
		
	% use ONLY \newpage if you want to force a page break for
	% ONLY the current column
	\newpage

\end{multicols*}
%\end{paracol}

\end{document}
